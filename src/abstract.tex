In genetic algorithms, the importance of the basis for representation has been well known.
In this paper, we emprirically studied the effect of a \textit{good basis} in binary representation, where a \textit{good basis} means a basis that can improve the serch performance of search algorithms.
A complicated problem space may be transformed into a linearly-separable one via a change of basis.
We had experiments on the one-max problem to clearly see the effect of basis change on search performance.
From these experiments, we could see that search performance is not so good with badly-designed genetic operators.
On the other hand, finding a good basis from all the bases may not be practical, because it takes time as $ O(2^{n^2}) $, where $ n $ is the length of a chromosome.
However, we used a genetic algorithm to find a good basis, to correctly inverstigate how a good basis affects the problem space.
We also conducted experiments on the $ NK $-landscape model as a representative computationally hard problem.
Experimental results showed that changing basis by the presented genetic algorithm always leads better search performance on the $ NK $-landscape model.


This paper provides a sample of a \LaTeX\ document which conforms,
somewhat loosely, to the formatting guidelines for
ACM SIG Proceedings.\footnote{This is an abstract footnote}
