In genetic algorithms, the importance of the basis for representation has been well known.
In this paper, we studied the effect of a \textit{good basis} in binary representation,
and resultantly we could show that a good basis improves the performance of search algorithms.
A complicated problem space may be transformed into a linearly-separable one via a change of basis.
We had experiments on search performance.
From these experiments, we could see that search performance is not so good with a badly-adopted basis.
On the other hand, finding a good basis from all the bases may not be practical,
because it takes $ O( 2^{n^2} ) $ time, where $ n $ is the length of a chromosome.
However, we used a genetic algorithm to find a good basis, to correctly investigate how a basis affects the problem space.
We also conducted experiments on the $ NK $-landscape model as a representative computationally hard problem.
Experimental results showed that changing basis by the presented genetic algorithm always leads better search performance on the $ NK $-landscape model.
